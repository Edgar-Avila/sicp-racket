% Created 2026-01-30 Fri 21:22
% Intended LaTeX compiler: pdflatex
\documentclass[11pt]{article}
\usepackage[utf8]{inputenc}
\usepackage[T1]{fontenc}
\usepackage{graphicx}
\usepackage{longtable}
\usepackage{wrapfig}
\usepackage{rotating}
\usepackage[normalem]{ulem}
\usepackage{amsmath}
\usepackage{amssymb}
\usepackage{capt-of}
\usepackage{hyperref}
\author{stack}
\date{\today}
\title{Notes on SICP}
\hypersetup{
 pdfauthor={stack},
 pdftitle={Notes on SICP},
 pdfkeywords={},
 pdfsubject={},
 pdfcreator={Emacs 29.3 (Org mode 9.6.15)}, 
 pdflang={English}}
\begin{document}

\maketitle
\tableofcontents


\section{Building Abstractions with Procedures}
\label{sec:org62d04d5}
\subsection{The elements of programming}
\label{sec:org9fe94b8}
We can combine simple ideas into a complex one using this:
\begin{itemize}
\item Primitive expressions: Simplest entities the language is concerned with
\item Means of combination: Compound elements built from simpler ones
\item Means of abstraction: How compound elements can be named and manipulated as units
\end{itemize}
\subsubsection{The substitution model}
\label{sec:org79a39e7}
The interpreter evaluates the element of the combination and applies
the procedure to the arguments. We can do this by two means. Assuming
the following procedures.
\begin{verbatim}
(define (square a) (* a a))
(define (sum-of-squares a b) (+ (square a) (square b)))
(define (f a) (sum-of-squares (+ a 1) (* a 2)))
\end{verbatim}
\subsubsection{Applicative order}
\label{sec:orgf01f153}
Evaluate the arguments then apply, the method the interpreter
actually uses.
\begin{verbatim}
;; When we evaluate
(f 5)
;; The reductions are as follows
(sum-of-squares (+ 5 1) (* 5 2))
(+ (square 6) (square 10))
(+ (* 6 6) (* 10 10))
(+ 36 100)
136
\end{verbatim}
\subsubsection{Normal order}
\label{sec:org7e8bddd}
Fully expand then reduce, an alternative method.
\begin{verbatim}
;; When we evaluate
(f 5)
;; We expand everything
(sum-of-squares (+ 5 1) (* 5 2))
(+ (square (+ 5 1)) (square (* 5 2)))
(+ (* (+ 5 1) (+ 5 1)) (* (* 5 2) (* 5 2)))
;; Then reduce
(+ (* 6 6) (* 10 10))
(+ 36 100)
136
\end{verbatim}
\subsection{Procedures and the processes they generate}
\label{sec:orge2ab651}
\subsubsection{Recursive processes}
\label{sec:org36cd94a}
Consider:
\begin{verbatim}
(define (factorial n)
  (if (= n 1)
      1
      (* n (factorial (n-1)))))
\end{verbatim}
If we apply the substitution model we end up with:
\begin{verbatim}
(factorial 4)
(* 4 (factorial 3))
(* 4 (* 3 (factorial 2)))
(* 4 (* 3 (* 2 (factorial 1))))
(* 4 (* 3 (* 2 1)))
(* 4 (* 3 2))
(* 4 6)
\end{verbatim}
We can observe the process expands and reduces as it's evaluating the
expressions, the state is held within the chain of deferred
operations, this is a \emph{recursive process}, since the number of
expansions and reductions grows linearly with n, this is called a
\emph{linear recursive process}.
\subsubsection{Iterative processes}
\label{sec:org2c9c521}
Now consider the following:
\begin{verbatim}
(define (factorial n)
  (fact-iter 1 1 n))
(define (fact-iter product counter max-count)
  (if (> counter max-count)
      product
      (fact-iter (* counter product)
                 (+ counter 1)
                 max-count))
\end{verbatim}
It grows like so:
\begin{verbatim}
(factorial 4)
(fact-iter 1 1 4)
(fact-iter 1 2 4)
(fact-iter 2 3 4)
(fact-iter 6 4 4)
(fact-iter 24 5 4)
24
\end{verbatim}
This does not grow and shrink, all the state is contained within the
arguments to the procedures, we could stop and resume the chain at any
point in time if we count with the right arguments to pass to the
parameters.
\subsubsection{Tree Recursion}
\label{sec:orgba00d20}
Consider:
\begin{verbatim}
(define (fib n)
  (cond ((=n 0) 0)
        ((= n 1) 1)
        (else (+ (fib (- n 1))
                 (fib (- n 2))))))
\end{verbatim}
It evolves like this
\begin{verbatim}
                               (fib 4)
                (+ (fib 3)               (fib 2))
        (+ (+ (fib 2) (fib 1))     (+ (fib 1) (fib 0)))
(+ (+ (+ (fib 1) (fib 0)) (fib 1)) (+ (fib 1) (fib 0)))
(+ (+ (+    1       0   )    1   ) (+    1       0))

\end{verbatim}
The process evolves into a tree of calls to different procedures, hence
it's name.
\subsubsection{Orders of Growth}
\label{sec:orge81547e}
\(R(n)\) has an order of growth \(R(n)=\Theta(f(n))\) if there are positive
constants \(k_1\) and \(k_2\) independent of n such that:
\begin{equation}
k_1f(n) \le R(n) \le k_2f(n)
\end{equation}

The order of growth only provides a crude description of the behavior
of the process, a process with \(n^2\) steps and a process with
\(1000n^2\) steps will all have \(theta(n^2)\) order of growth.

\subsection{Formulating abstractions with higher-order procedures}
\label{sec:org20bd01a}
\subsubsection{Procedures as arguments}
\label{sec:org8dd0f54}
Procedures that manipulate procedures are called higher-order procedures,
they are useful when the same programming pattern is used
with a number of different procedures.
For example, the following can abstract the pattern of "summing"
numbers, and we can use it to create new procedures that
sum cubes, integers, etc.
\begin{verbatim}
(define (identity (a) a))
(define (square a) (* a a))
(define (cube a) (* a a a))
(define (1+ a) (+ a 1))
(define (sum term a next b)
  (if (> a b)
      0
      (+ (term a)
         (sum term (next a) next b)))
(define (sum-cubes a b)
  (sum cube a 1+ b))
(define (sum-squares a b)
  (sum square a 1+ b))
(define (sum-ints a b)
  (sum identity a 1+ b))
\end{verbatim}
\subsubsection{Procedures using lambda}
\label{sec:org0f948e5}
For trivial procedures, it's often more convenient to directly
specify them without any names, rather than defining them. This
is what lambdas are for, really useful when passing them as
arguments to higher-order procedures.
\begin{verbatim}
(define (sum-cubes a b)
  (sum (lambda (x) (* x x x)) a 1+ b))
(define (sum-squares a b)
  (sum (lambda (x) (* x x) a 1+ b))
(define (sum-ints a b)
  (sum (lambda (x) x) a 1+ b))
\end{verbatim}
\subsubsection{Using let to create local variables}
\label{sec:org04f87f4}
Imagine we wanted to express:
\[f(x,y)=x(1+xy)^2+y(1-y)+(1+xy)(1-y)\]
Which could be also expressed as:
\[a=1+xy\]
\[b=1-y\]
\[f(x,y)=xa^2+yb+ab\]
We could do:
\begin{verbatim}
(define (f x y)
  (define (f-helper a b)
    (+ (* x (square a))
       (* y b)
       (* a b)))
  (f-helper (+ 1 (* x y))
            (- 1 y)))
\end{verbatim}
Which binds a and b to the computed values before applying
the rest of the procedure, but it's somewhat inconvenient,
for this, we can use the special form called \emph{let}.
\begin{verbatim}
(define (f x y)
  (let ((a (+ 1 (* x y)))
        ( b (- 1 y)))
    (+ (* x (square a))
       (* y b)
       (* a b))))
\end{verbatim}
\subsubsection{Finding roots of equations by the half-interfal method}
\label{sec:orgecaf9b6}
The half-interval method is a technique for finding the roots
of an equation \(f(x)=0\), where \(f\) is a continuous function and
we count with points \(a\) and \(b\) such that \(f(a) < 0 < f(b)\),
we do this by averaging \(a\) and \(b\) many times, reducing
the interval we're searching on each time. (We're basically binary
searching).
\begin{verbatim}
;; Helper for the tolerance
(define (close-enough? x y)
  (< (abs (- x y)) 0.001))
;; Actual search
(define (search f neg-point pos-point)
  (let ((midpoint (average neg-point pos-point)))
    (if (close-enough? neg-point pos-point)
        midpoint
        (let ((test-value (f midpoint)))
          (cond ((positive? test-value)
                 (search f neg-point midpoint))
                ((negative? test-value)
                 (search f midpoint pos-point))
                (else midpoint))))))
;; Checks if they're actually of opposite signs
;; and picks on which value passed is the negative
;; one and the positive one in order to pass
;; them correctly to search
(define (half-interval-method f a b)
  (let ((a-value (f a))
        (b-value (f b)))
    (cond ((and (negative? a-value) (positive? b-value))
           (search f a b))
          ((and (negative? b-value) (positive? a-value))
           (search f b a))
          (else
           (error "Valeus are not of opposite sign" a b)))))
;; We can then use it to approximate pi
;; (pi is the point where sin(x) = 0)
(half-interval-method sin 2.0 4.0) ;; 3.14111328125
\end{verbatim}
\subsubsection{Finding fixed points of functions}
\label{sec:orgc5558c1}
A number \(x\) is the \emph{fixed point} of a function if \(f(x) = x\), for
some functions we can find this with an initial guess and applying \(f\)
repeatedly until the value does not change very much (a tolerance)
\[f(x), f(f(x)), f(f(f(x))), ...,\]
\begin{verbatim}
(define tolerance 0.00001)
(define (fixed-point f first-guess)
  (define (close-enough? v1 v2)
    (< (abs (- v1 v2))
       tolerance))
  (define (try guess)
    (let ((next (f guess)))
      (if (close-enough? guess next)
          next
          (try next))))
  (try first-guess))
;; Trying it on cos
(fixed-point cos 1.0)
;; We can even define a sqrt function in terms of this
(define (sqrt x)
  (fixed-point (lambda (y) (average y (/ x y)))
               1.0))
\end{verbatim}
\end{document}
